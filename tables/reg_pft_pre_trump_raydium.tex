\begin{tabular}{lcccc}
\hline
 & \multicolumn{4}{c}{$\text{Profit}$} \\
 $\text{Bot}:$ & $\text{Rat Bot}_{i}$ & $\text{Bundle Bot}_{i}$ & $\text{Wash Trading Bot}_{i}$ & $\text{Comment Bot}_{i}$ \\
 & (1) & (2) & (3) & (4)\\
\hline
$\text{Creator}$ & 1711.41*** & 3670.89*** & 2229.73*** & 2260.30*** \\
 & (37.99) & (56.77) & (42.25) & (38.43) \\
$\text{Creator} \times \text{Bot}$ & 4722.01*** & -44.88*** & -574.43*** & -9.04*** \\
 & (135.37) & (1.23) & (83.69) & (0.62) \\
$\text{Non-Creator}$ & -4.36*** & -5.01*** & -6.31*** & -4.86*** \\
 & (0.77) & (1.11) & (1.17) & (0.82) \\
$\text{Non-Creator} \times \text{Bot}$ & -1.39 & 0.02 & 3.06** & 0.02 \\
 & (2.73) & (0.02) & (1.51) & (0.01) \\
$\text{Observations}$ & 2371204 & 2371204 & 2371204 & 2371204 \\
$R^2$ & 0.00 & 0.00 & 0.00 & 0.00 \\
\hline
\end{tabular}